
\section{Figures de la transcendance}

Il va s'agir de voir, de suivre, d'essayer de se repérer dans les différentes mutations du concept philosophique de transcendance au travers du temps.

Il va non seulement s'agir de se repérer dans l'histoire d'un concept, mais également essayer de voir ce concept, de se rendre capable de décrire ce que signifie la transcendance dans une perspective \textbf{phénoménologique}, d'accéder à une description possible de ce que les différents concepts de transcendance recoupent.

\subsection{Première définition}

La transcendance signifie un dépassement : plus que tout autre catégorie métaphysique, la transcendance semble davantage résister à toute tentative de description. Comment ce qui par définition est au delà se donne-t-il ? Comment cela nous dépasse ?

Nous allons nous demander dans quelle mesure la transcendance nous présente des figures, en dépit du dépassement, du lointain qu'elle suggère : nous allons nous demander comment la phénoménologie peut accueillir et déployer le concept de transcendance.


\section{Le concept de transcendance}


Nous allons voir en quoi la question de la transcendance est le cœur de la métaphysique.

La transcendance comporte trois moments descriptifs : une destination, une provenance et l'action elle-même d'outre-passer : c'est un processus dynamique. Immédiatement, on se rend compte de la symétrie de ces trois moments dans le sens où deux d'entre eux sont déjà déterminés quand le troisième demeure encore indéterminé, obscur : la destination de la transcendance, indépendamment de toute caractérisation extérieure, est ce qui est supérieur.


\subsection{La destination de la transcendance}
Ce que vise la transcendance, c'est ce qu'il y a de plus haut : ce n'est pas forcément ce qui est autre, ce en deça de quoi le reste se tient. La destination de la transcendance, quel que soit son nom, qu'on lui donne ou pas, est toujours \textbf{le plus haut}. En ce sens là, c'est l'\textbf{indépassable}, le plus haut. Poser la question de la transcendance, c'est poser la question de l'orientation dernière de toute élévation.

Ainsi, nécessairement, dans cette question, on voit en quel sens la question métaphysique se trouve toujours comprise sous une interrogation éthique en ce sens que la transcendance indique toute élévation, tout dépassement vers le haut que chacun peut engager pour soi-même. Si l'éthique est, dans un sens platonicien, la recherche de l'excellence, une telle éthique ne peut pas ne pas être intriquée dans la question métaphysique, ontologique, vers ce qui est la mesure de toute hauteur, de toute excellence.

Cette question métaphysique particulière, n'est jamais une question indifférente pour celui qui la pose, qu'on pourrait poser objectivement, sans y prendre le moindre intérêt : elle nous concerne, nous interpelle, nous donne notre propre mesure.


\subsection{Le mouvement}

Ce mouvement de transcendance est caractéristique : il n'est pas un mouvement horizontal mais vertical. Il s'agit d'aller vers ce qui surpasse, faire l'expérience de la transcendance, veut dire nécessairement faire l'expérience d'une verticalité, d'un plus haut, monter d'une certaine façon.


\subsection{La provenance}

La provenance de la transcendance, en revanche, est le plus obscur, ce d'où provient la transcendance, ce qui est inférieur, ouvre déjà une première interrogation, responsable des différents sens que la transcendance aura reçu dans l'histoire de la philosophie. Plus simplement : que transcende-t-on ?

C'est une question métaphysique, où on peut voir un parallèle avec le terme transcendance en latin et la métaphysique, et ce serait alors penser ce qui est plus haut et supérieur à la nature, sur-naturel. Le transcendant sera alors le supra-sensible, ce qui est au dessus du sensible, ça ne veut pas dire ce qui n'est soi même pas sensible, mais qu'il est aussi ce dont la nature sensible dépend. Ce n'est pas seulement ce qui ne dépend pas de l'expérience sensible, mais est présupposé par le sensible, ce n'est pas détaché, hors du sensible, mais le support même du sensible


